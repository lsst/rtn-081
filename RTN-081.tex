
\documentclass[]{spie}

% Package imports go here.
\renewcommand{\baselinestretch}{1.0} % Change to 1.65 for double spacing

\usepackage{amsmath,amsfonts,amssymb}
\usepackage{graphicx}
\usepackage[colorlinks=true, allcolors=blue]{hyperref}
\usepackage{listings}
\usepackage{xcolor}
\usepackage{longtable}

% Local commands go here.
\newcommand{\aj}{AJ}
\newcommand{\apj}{ApJ}
\newcommand{\apjs}{ApJS}
\newcommand{\procspie}{Proc.\ SPIE}
\newcommand{\pasj}{PASJ}
% lsstdoc documentation: https://lsst-texmf.lsst.io/lsstdoc.html
\input{meta}

% Package imports go here.

% Local commands go here.

% See ASPmanual2010.pdf 2.1.4  and ManuscriptInstructions.pdf for more details
%\markboth{auth}{short title}


\newcommand{\docRef}{RTN-081}
\newcommand{\docUpstreamLocation}{\url{https://github.com/lsst/rtn-081}}



\begin{document}
\input{authors}
\title{Rubin Observatory Operations: Enabling collaborative ground-up budget planning across a multi-team organization}

% This can write metadata into the PDF.
% Update keywords and author information as necessary.
\hypersetup{
    pdftitle={Rubin Observatory Operations: Enabling collaborative ground-up budget planning across a multi-team organization},
    pdfauthor={petryce},
    pdfkeywords={Vera C. Rubin Observatory, budget, planning, funding}
}

\maketitle


\begin{abstract}
Change is inevitable in large big budget operational programs. Embracing, rather than resisting, change is key to being proactive. It also keeps teams motivated as it’s another avenue for leadership to “listen” to what is going on at the team level. At Rubin Observatory an agile approach to budgeting has been implemented. Annually a bottom up review, called a scrub, is carried out across all departments of the Rubin Operations organization providing an opportunity to adapt and be nimble to changing situations that can affect resources and budgets. This paper will provide details on the importance for an annual budget scrub, processes followed, tools used and how the cycle continues year on year.
\end{abstract}



\keywords{Vera C. Rubin Observatory, budget, planning, funding}


\section{Introduction} \label{sec:intro}
Vera C. Rubin Observatory\cite{2008arXiv0805.2366I} is currently under construction, once complete the observatory will consist not only of the  physical Chile based mountain top summit facility housing an 8.5m telescope and a 3200 megapixel camera but also includes systems such as those for data management and public engagement. Naturally this means the observatory is distributed across a number of locations. The operations program is due to start full operations in 2025, the survey will be carried out over a period of ten years and a post-operations phase will follow. The operation of Rubin observatory is fairly unique in being funded in equal shares by two US government agencies. Budgets are usually set at high levels years in advance but things rarely stay the same. This paper describes an annual bottom up budgeting process adapted from the one used by the ATLAS Experiment. The aim of the annual exercise is to enable change within the budget envelope.
The paper provides the reader with details of all the tools that are used and describes the process that is followed annually.

\section{Tools} \label{sec:tools}
text here

\section{Process} \label{sec:process}


Throughout the Rubin pre-operations and operations phases, annually in May each team will look back at what was planned, what was achieved, do a full review of its activities, and propose a high level plan for the following (US fiscal) year.
This is standard practice in other high energy physics experiments as well, the scrub allows the facility to continuously evolve its operating plan, taking critical input from the people that understand best what is really needed, in Rubin’s case that is the Team Leaders.
Following the National Science Foundation and Department of Energy joint annual review of Rubin Operations the Rubin Operations Directors office together with department heads sets the major milestones for the next US fiscal year (FY) starting 1st October. This includes looking at the status of major milestones for the current year and ascertaining whether any of those need to carry over into the next FY.
With the major milestones set, the Director’s office kicks off the month long annual scrub process, see \autoref{fig:timeline} in which the department heads start down stream planning with their teams. This is the “homework” phase of the scrub where teams are looking at:

\begin{itemize}
\item status of minor milestones for the current FY
\item setting minor milestones that would contribute to accomplishing the new major milestones for the next FY
\item based on activities needed to achieve the minor milestones and risk mitigation plans the teams review planned resources both labor and non-labor
\item if there is a mismatch between resources needed and the resources available the team will propose changes during this scrub period through the tool (described in the next section).

\end{itemize}


\begin{figure}
\begin{centering}
\includegraphics[width=0.9\textwidth]{Figure1Scrubtimeline}
	\caption{Scrub timeline
\label{fig:timeline}}
\end{centering}
\end{figure}

\section{Outcomes and Conclusions} \label{sec:outcomes}

This annual iterative process enables change to happen in a controlled and transparent manner enabling buy-in at all levels on what the upcoming FY priorities are and the reasons behind difficult decisions which are often inevitable. It should be noted that changes can still happen throughout the year through a process called Request Beyond Target (RBT). This process is outside of the scope of this paper but is mentioned here to stress the agile nature of planning on the Rubin Program.

Year on year, as this process takes place from now through to the end of the ten-year Legacy Survey of Space and Time, expected in 2037, the scrub process is envisaged to evolve. Each iteration is expected to reveal gaps and areas of improvement that can fed into the design of the process and the tools for the following fiscal year’s scrub.





\acknowledgments
This material or work is supported in part by the U.S. National Science Foundation through Cooperative Agreement AST-1258333 and Cooperative Support Agreement AST2211468 managed by the Association of Universities for Research in Astronomy (AURA), and the U.S. Department of Energy under Contract No. DE-AC02-76SF00515 with the SLAC National Accelerator Laboratory managed by Stanford University.

Thank you to Srini Rajagopalan and the US-ATLAS team for sharing this scrub process and for allowing members of the Rubin team to observe it in action during 2022.

% Include all the relevant bib files.
% https://lsst-texmf.lsst.io/lsstdoc.html#bibliographies
\bibliographystyle{spiebib}
\bibliography{local,lsst,lsst-dm,refs_ads,refs,books}
\pagebreak
% Make sure lsst-texmf/bin/generateAcronyms.py is in your path
\section*{Acronyms} \label{sec:acronyms}
\input{acronyms.tex}


\end{document}
