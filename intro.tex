\section{Introduction} \label{sec:intro}
\cite{2008arXiv0805.2366I}
Vera C. Rubin Observatory is currently under construction, once complete the observatory will consist not only of the  physical Chile based mountain top summit facility housing an 8.5m telescope and a 3200 megapixel camera but also includes systems such as those for data management and public engagement. Naturally this means the observatory is distributed across a number of locations. The operations program is due to start full operations in 2025, the survey will be carried out over a period of ten years and a post-operations phase will follow. The operation of Rubin observatory is fairly unique in being funded in equal shares by two US government agencies. Budgets are usually set at high levels years in advance but things rarely stay the same. This paper describes an annual bottom up budgeting process adapted from the one used by the ATLAS Experiment. The aim of the annual exercise is to enable change within the budget envelope.
The paper provides the reader with details of all the tools that are used and describes the process that is followed annually.
