\section{Introduction} \label{sec:intro}
Vera C. Rubin Observatory\cite{2008arXiv0805.2366I} is currently under construction.
Once complete, the observatory will consist of an end to end system with the mountain top Summit Facility on Cerro Pachon in Chile housing an 8.4m telescope and a 3200 megapixel camera, a high bandwidth long haul network, a system of data processing facilities in California, France and the UK, a data management system, a data access platform in the cloud, and a host of public engagement programs.
Naturally this means the observatory is distributed across a number of locations.

The observatory is due to start full operations in 2025; the survey will be carried out over a period of ten years, and a post-operations phase will follow.
The operation of Rubin Observatory is fairly unique in astronomy for being funded in approximately equal shares by two US government agencies, and operated by two almost equal partner national laboratories (NSF NOIRLab and SLAC National Accelerator Laboratory, funded by the Department of Energy).
Budgets are usually set at high level years in advance, but things rarely stay the same.

This paper describes an annual ground-up budgeting process adapted from the one used by the US-ATLAS operations team, in support of the ATLAS experiment at the Large Hadron Collider at CERN \cite{US-ATLAS,Srini}.
The aim of this annual exercise is to enable change within the budget envelope.
The paper provides the reader with details of all the tools that are used, and describes the process that is followed annually.
